%%---------------------------------------------------------------------------%%
%%------------ 第一章:绪论 -------------------------------------------------%%
%%---------------------------------------------------------------------------%%


\chapter{绪论}
\label{chapter:Introduction}


\section{研究背景}











\section{迭代法介绍}

\begin{itemize}
\item
一般的~Newton~法~(初始点~$x_0$~ 给定):
$$
F'(x_k) \Delta x_k = -F(x_k),\ x_{k+1} = x_k + \Delta x_k, \quad k =
0,1,\ldots.
$$
\item
简化~Newton~法:
这种变形的~Newton~法将每步计算~$F'(x_k)$~改为固定的~$F'(x_0)$~(初始点~$x_0$~
给定):
$$
F'(x_0) \Delta x_k = -F(x_k),\ x_{k+1} = x_k + \Delta x_k,\quad k =
0,1,\ldots.
$$
这样, 每步只需要计算~$n$~个分量函数, 但这种迭代法只有线性收敛.
\item
~Newton~类法: 当所要处理的方程组维数较大时,
很难求解~Newton~方程~(\ref{eq:NewtonEquation})~的精确解~$\Delta
x_k$, 进而无法得到精确的~Jacobi~矩阵~$F'(x_{k+1})$.
为了能够应用~Newton~法较好的处理这种情况,
得到了如下的一种变形~Newton~法~(初始点~$x_0$~ 给定):
\begin{equation}
\label{it:NML_general} M(x_k)\Delta x_k = -F(x_k),\ x_{k+1} = x_k +
\Delta x_k,\ \ k = 0,1,\ldots,
\end{equation}
其中~$M(x_k)$~是近似于~$F'(x_k)$~的矩阵.
\item
非精确~Newton~法: 同样考虑大规模方程组情形,
相比于~Newton~类法用一个近似的~Jacobi~矩阵来代替~$ F'(x_k)$,
若考虑在求解~Newton~方程~(\ref{eq:NewtonEquation})~时,
不去求其精确解而只需要求满足某种条件的近似解来作为迭代修正,
这样便得到如下的变形~Newton~法~(初始点~$x_0$~ 给定):
\begin{equation}
\label{it:INM_BanachSpace} F'(x_k) \Delta x_k = -F(x_k) + r_k,\
x_{k+1} = x_k + \Delta x_k,\quad k = 0,1,\ldots,
\end{equation}
其中~$r_k \in \mathbb{Y}$~一般应满足~$\|r_k\|/\|F(x_k)\|\leqslant
\eta_k,\ k= 0,1,\ldots$~, $\{\eta_k\}$~满足~$0\leqslant \eta_k <1$,
可能与~$x_k$~有关, 为控制序列,
用来控制求方程~(\ref{eq:NewtonEquation})~的解的精确程度.
显然令~$\eta_k \equiv 0$~时得到一般的~Newton~法.
\item
拟~Newton~法:Newton~法的主要缺点之一是每步都要计算导数~$F'(x)$~的值,
当分量函数较复杂时计算很不方便,
拟~Newton~法是针对这一缺点提出的算法,
其核心是用通过计算函数值来代替导数以避免求导:
$$
J_k \Delta x_k = -F(x_k), \ J_{k+1} = J_k + \Delta J_k,\quad k =
0,1,\ldots,
$$
其中~$J_k$~是近似于~$F'(x_k)$~的矩阵.不同的~$\Delta J_k$~选择,
可以得到不同的迭代法. 例如,当取$\Delta J_k = F(x_{k+1})\tran{\Delta
x_k}/(\tran{\Delta x_k}\Delta x_k)$, 则得到~Broyden~法。
\item
~Gauss-Newton~法:
这种变形的~Newton~法主要应用于求解非线性最小二乘~(约束/无约束)~问题,
迭代格式为:
\begin{equation}
\label{it:GNM_BanachSpace} \|F'(x_k)\Delta x_k + F(x_k)\| = \min,\
x_{k+1} = x_k + \Delta x_k,\quad k = 0,1,\ldots.
\end{equation}
\end{itemize}




\begin{itemize}
\item[1)]
Kantorovich~型收敛理论:理论上,
Newton~法收敛性的一个最重要收敛结果是被称为
~Newton-Kantorovich~半局部收敛定理~\cite{Kantorvich1982}.
该定理在理论和应用上都是相当重要的, 它是解方程算法现代研究的起点.
大量的收敛结果都是基于所谓的~Kantorovich~型条件而得到的,
例如,\cite{Ortega1970,Rokne1972,Rall1974,GraggTapia1974,Deuflhard1979,
Ypma1982,Huang1993,Gutierrez1997a,Wang1999,Gutierrez2000,Ezquerro2002}.

\item[2)]
Smale~点估计理论:该理论是由~Smale~于~1986~年提出,由~$\alpha-$理论
和~$\gamma-$理论组成。在~$\alpha-$理论中,
假设~$F$~在初始点~$x_0$~是解析的,
给出了基于如下三个不变量的收敛判据~\cite{Smale1986}:
\begin{equation*}
\begin{cases}
\alpha(F,x_0) = \beta(F,x_0)\gamma(F,x_0),\\
\beta(F,x_0) = \|F'(x_0)^{-1}F(x_0)\|,\\
\gamma(F,x_0) = \sup\limits_{k \geqslant 2} \left\|\displaystyle
\frac{1}{k!} F'(x_0)^{-1}F^{(k)}(x_0)^{-1}\right\|^{\frac{1}{k-1}}.
\end{cases}
\end{equation*}
而~$\gamma-$理论则研究了算子~$F$~在解析条件下的局部收敛性。
定理~\ref{th:SmaleGammaTh}~称为~$\gamma-$定理。
王兴华等人改进并完善了~Smale~点估计理论
~(见文献~\cite{WangLi2001}~及其所列文献). 值得指出的是,
王兴华引入了~$\gamma$~条件
并在此基础上系统建立了~Smale~原先在解析条件下的全部结果
(见文献~\cite{WangHan1997b}~及其所列文献).
\end{itemize}






\section{矩阵函数}


\begin{theorem}[{\cite[定理~1.15]{Higham2008}}]
\label{th:MatFun_sum_product} 设函数~$f,g$~在矩阵~$A\in\CS^{n\times
n}$~的谱上有定义.
\begin{enumerate}
\item[\textup{(i)}]
若~$h(t) = f(t) + g(t)$, 则~$h(A) = f(A)+g(A)$;
\item[\textup{(ii)}]
若~$h(t)=f(t)g(t)$, 则~$h(A)=f(A)g(A)$.
\end{enumerate}
\end{theorem}






\subsection{矩阵平方根}

\begin{theorem}[\cite{HornJohnson1991}]

\end{theorem}

\begin{theorem}[矩阵平方根的分类, \cite{Higham2008}]

\end{theorem}



\begin{theorem}[\cite{Higham2008}]
\end{theorem}


\begin{align*}
u_{ii}^2 & = t_{ii},\quad i = 1,2,\ldots,n,\\
(u_{ii}+u_{jj})u_{ij} & = t_{ij} - \sum_{k=i+1}^{j-1}u_{ik}u_{kj},
\quad j > i.
\end{align*}
于是, 有如下计算非奇异矩阵平方根的算法,
该算法由~Bj\"{o}rck~$\&$~Hammarling~于~\cite{Bjorck1983}~得到.

\begin{algorithm}[h!]
\floatname{algorithm}{算法}
\caption{计算矩阵平方根的~Schur~法~\cite{Bjorck1983}}
\label{al:MatSquRoot_SchurMethod} 给定非奇异矩阵~$A\in\CS^{n\times
n}$, 本算法通过~Schur~分解来计算~$A$~的主平方根~$A^{1/2}$.
\newcounter{newlist}
\begin{list}{\arabic{newlist}.}{\usecounter{newlist}
\setlength{\rightmargin}{0em}\setlength{\leftmargin}{1.2em}}
\item
计算矩阵~$A$~的~Schur~分解~$A = QRQ^*$;
\item
计算矩阵~$U$~各对角元素的主平方根~$u_{ii} = t_{ii}^{1/2},\ i = 1,
\ldots, n$;
\item
依次计算矩阵~$U$~的非对角元:
$$
u_{ij} = \displaystyle
\frac{t_{ij}-\displaystyle\sum_{k=i+1}^{j-1}u_{ik}u_{kj}}{u_{ii}+u_{jj}},
\quad j = 2,3,\ldots,n,\ i = j-1,j-2,\ldots,1;
$$
\item
计算~$X = QUQ^*$.
\end{list}
\end{algorithm}

\begin{algorithm}[h!]
\floatname{algorithm}{算法}
\caption{计算矩阵平方根的实~Schur~法~\cite{Higham1987}}
\label{al:MatSquRoot_RealSchurMethod} 给定矩阵~$A\in\RS^{n\times
n}$, 其所有特征值都不属于~$\RS^- := (-\infty,0]$.
本算法通过实~Schur~分解来计算~$A$~的主平方根~$A^{1/2}$.
\begin{list}{\arabic{newlist}.}{\usecounter{newlist}
\setlength{\rightmargin}{0em}\setlength{\leftmargin}{1.2em}}
\item
计算矩阵~$A$~的实~Schur~分解~$A = QR\tran{Q}$, 其中~$R$~是~$m\times
m$~的块矩阵.
\item
计算矩阵~$U$~各对角块的主平方根:当~$R_{ii}=[r_{ij}]_{1\times1}$~时,
$U_{ii} = R_{ii}^{1/2}$;当~$R_{ii}=[r_{ij}]_{2\times2}$~时,
$$
U_{ii} = \left[\begin{array}{cc} \alpha +
\displaystyle\frac{1}{4\alpha}(r_{11}-r_{22}) &
\displaystyle\frac{1}{2\alpha}r_{12} \\
\displaystyle\frac{1}{2\alpha}r_{21} & \alpha -
\displaystyle\frac{1}{4\alpha}(r_{11}-r_{22})
\end{array}\right],
$$
其中
\begin{equation*}
\alpha = \left\{
\begin{array}{ll}
\displaystyle \left(\frac{|\theta|+(\theta^2 +
\mu^2)^{1/2}}{2}\right)^{1/2}, & \theta \geq 0,\\
\displaystyle
\frac{\mu}{2\left(\displaystyle\frac{|\theta|+(\theta^2 +
\mu^2)^{1/2}}{2}\right)^{1/2}}, & \theta <0,
\end{array}
\right.
\end{equation*}
$$
\theta = \frac{r_{11}+r_{12}}{2},\quad \mu =
\frac{\left(-(r_{11}-r_{22})^2-4r_{21}r_{22}\right)^{1/2}}{2}.
$$
\item
依次通过计算如下的方程而得到矩阵~$U$~的非对角块~$U_{ij}$:
$$
U_{ii}U_{ij} + U_{ij}U_{jj} = R_{ij} - \sum_{k=i+1}^{j-1}
U_{ik}U_{kj}, \quad j = 2,3,\ldots,m,\ i = j-1,j-2,\ldots,1.
$$
\item
计算~$X = QU\tran{Q}$.
\end{list}
\end{algorithm}




\begin{lemma}[{\cite[引理~6.8]{Higham2008}}]
假设~Newton~法~$(\ref{it:NM_MatSquRoot_original})$~
的初始点~$X_0$~与矩阵~$A$~可交换,
且所产生的序列~$\{X_k\}$~是有定义的. 则对于所有的~$k\geq1$,
$X_k$~与~$A$~都是可交换的,  且此时~Newton~法的迭代格式为:



\subsection{矩阵~$p$~次根}




\begin{algorithm}[h!]
\floatname{algorithm}{算法}
\caption{计算矩阵主~$p$~次根的~Schur-Newton~法~\cite[算法~
3]{Iannazzo2008}} \label{al:MatpthRoot_SchurNewton_Ian08}
给定矩阵~$A\in\RS^{n\times n}$, 其所有特征值都不属于~$\RS^- :=
(-\infty,0]$. 给定整数~$p\geq 2$, 则存在整数~$k_0 \geq
0$~及奇数~$q$~使得~$p = 2^{k_0}q$.
本算法通过实~Schur~分解和对偶~Newton~法~(\ref{it:NM_MatpthRoot_Coupled})~
来计算~$A$~的主~$p$~次根~$A^{1/p}$.
\begin{list}{\arabic{newlist}.}{\usecounter{newlist}
\setlength{\rightmargin}{0em}\setlength{\leftmargin}{1.2em}}
\item
计算矩阵~$A$~的实~Schur~分解~$A = QR\tran{Q}$.
\item
若~$q=1$,令~$k_1 = k_0$; 若~$q\neq1$,则选取~$k_1\geq k_0$~使得
存在正数~$s$~使任意~$A$~的特征值~$\lambda$~满足
$$
s \lambda^{1/{2^{k_1}}} \in \left\{z\in \CS: \left|z -
\frac{6}{5}\right| \leq \frac{3}{4}\right\}.
$$
\item
通过算法~\ref{al:MatSquRoot_RealSchurMethod}~计算~$B =
R^{1/{2^{k_1}}}$.
\item
若~$q=1$, 则令~$X = QB\tran{Q}$;若~$q\neq 1$,
则通过对偶~Newton~法~(\ref{it:NM_MatpthRoot_Coupled})~ 来计算~$C =
(B/s)^{1/q}$~并令~$X = Q(Cs^{1/q})^{2^{k_1-k_0}}\tran{Q}$.
\end{list}
\end{algorithm}



\begin{algorithm}[h!]
\floatname{algorithm}{算法}
\caption{计算矩阵主~$p$~次根的含参数~Schur-Newton~法~\cite[算法~
3.3]{GuoHigham2006}} \label{al:MatpthRoot_ParSchurNewton_GuoHig06}
给定矩阵~$A\in\RS^{n\times n}$, 其所有特征值都不属于~$\RS^- :=
(-\infty,0]$. 给定整数~$p\geq 2$, 则存在整数~$k_0 \geq
0$~及奇数~$q$~使得~$p = 2^{k_0}q$.
本算法通过实~Schur~分解和含参数~Newton~法~
(\ref{it:NM_MatpthRoot_ParCoupled})~
来计算~$A$~的主~$p$~次根~$A^{1/p}$.
\begin{list}{\arabic{newlist}.}{\usecounter{newlist}
\setlength{\rightmargin}{0em}\setlength{\leftmargin}{1.2em}}
\item
计算矩阵~$A$~的实~Schur~分解~$A = QR\tran{Q}$.
\item
若~$q=1$,令~$k_1 = k_0$; 若~$q\neq1$,则选取~$k_1\geq k_0$~使得
$|\lambda_1/\lambda_n|^{1/2^{k_1}}\leq 2$, 其中~$\lambda_1,\ldots,
\lambda_n$~为~$A$~的特征值且满足~$|\lambda_n|\leq \cdots \leq
|\lambda_1|$,
当~$\lambda_\ell$~不全是实数时,重新选取~$k_1$~使得对任意的~$\ell
\in \{1,2,\ldots,n\}$~都有
$$
\arg(\lambda_\ell^{1/2^{k_1}}) \in
\left(-\frac{\pi}{8},\frac{\pi}{8}\right).
$$
\item
通过算法~\ref{al:MatSquRoot_RealSchurMethod}~计算~$B =
R^{1/{2^{k_1}}}$.
\item
若~$q=1$, 则令~$X = QB\tran{Q}$;若~$q\neq 1$, 则先选取参数~$c$,
再通过含参数对偶~Newton~法~ (\ref{it:NM_MatpthRoot_ParCoupled})~
来计算~$C = B^{1/q}$~并令~$X = QC^{2^{k_1-k_0}}\tran{Q}$.
\end{list}
\end{algorithm}


\begin{algorithm}[h!]
\floatname{algorithm}{算法}
\caption{计算矩阵主~$p$~次根的~Schur-Halley~法~\cite[算法~
4]{Iannazzo2008}} \label{al:MatpthRoot_SchurHalley_Ian08}
给定矩阵~$A\in\RS^{n\times n}$, 其所有特征值都不属于~$\RS^- :=
(-\infty,0]$. 给定整数~$p\geq 2$, 则存在整数~$k_0 \geq
0$~及奇数~$q$~使得~$p = 2^{k_0}q$.
本算法通过实~Schur~分解和对偶~Halley~法~(\ref{it:HM_MatpthRoot_Coupled})~
来计算~$A$~的主~$p$~次根~$A^{1/p}$.
\begin{list}{\arabic{newlist}.}{\usecounter{newlist}
\setlength{\rightmargin}{0em}\setlength{\leftmargin}{1.2em}}
\item
计算矩阵~$A$~的实~Schur~分解~$A = QR\tran{Q}$.
\item
若~$q=1$,令~$k_1 = k_0$; 若~$q\neq1$,则选取~$k_1\geq k_0$~使得
存在正数~$s$~使任意~$A$~的特征值~$\lambda$~满足
$$
s \lambda^{1/{2^{k_1}}} \in \left\{z\in \CS: \left|z -
\frac{8}{5}\right| \leq 1\right\}.
$$
\item
通过算法~\ref{al:MatSquRoot_RealSchurMethod}~计算~$B =
R^{1/{2^{k_1}}}$.
\item
若~$q=1$, 则令~$X = QB\tran{Q}$;若~$q\neq 1$,
则通过对偶~Halley~法~(\ref{it:HM_MatpthRoot_Coupled})~ 来计算~$C =
(B/s)^{1/q}$~并令~$X = Q(Cs^{1/q})^{2^{k_1-k_0}}\tran{Q}$.
\end{list}
\end{algorithm}










\section{论文的组织}
