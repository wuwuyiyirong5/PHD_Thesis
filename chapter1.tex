%%---------------------------------------------------------------------------%%
%%------------ 第一章:绪论 -------------------------------------------------%%
%%---------------------------------------------------------------------------%%


\chapter{绪论}
\label{chapter:Introduction}


\section{研究背景}











\section{迭代法介绍}

令~$\mathbb{X}$~和~$\mathbb{Y}$~是欧氏空间或一般的~Banach~空间,
$\mathbb{D}$~是~$\mathbb{X}$~的一个开凸子集, 设~$F:\mathbb{D}
\subset \mathbb{X} \to
\mathbb{Y}$~是一个~Fr\'{e}chet~可导的非线性算子,
考虑如下一般的非线性算子方程:
\begin{equation}
\label{eq:NonlinearOperatorEquation} F(x) = 0.
\end{equation}
求解非线性算子方程~(\ref{eq:NonlinearOperatorEquation})~
的近似解是一个重要的数学问题.有别于线性方程组的情形,
求解非线性算子方程~(\ref{eq:NonlinearOperatorEquation})~
一般应用迭代方法. 目前, Newton~ 法是求解非线性算子方程
~(\ref{eq:NonlinearOperatorEquation})~的最有效方法,
其迭代格式定义为~(初始点~$x_0$~ 给定):
\begin{equation}
\label{it:NM_BanachSpace} x_{k+1} = x_k - F'(x_k)^{-1}F(x_k),\quad k
= 0,1,2,\ldots.
\end{equation}

用迭代法求解算子方程~(\ref{eq:NonlinearOperatorEquation})~,
基本的途径是构造一个有效的迭代格式
(如~Newton~法~(\ref{it:NM_BanachSpace})), 使得由给定的初始点出发,
逐步逼近到方程~(\ref{eq:NonlinearOperatorEquation})~的一个解. 于是,
迭代法的收敛性成为研究的一个核心问题. 一般情况下,
收敛性分析有以下三种类型:
\begin{itemize}
\item[1)]
局部收敛性:该类型首先假定方程
~(\ref{eq:NonlinearOperatorEquation})~ 存在解~$x^*$~,再
根据~$F$~在~$x^*$~的局部条件 ~(例如,$F$~在~$x^*$~是连续可微的)~
来研究有关迭代法的收敛性质,
其中包括收敛速度、解的唯一性球及~(最重要的)~收敛球半径的最优性.
例如,关于 ~Newton~法~(\ref{it:NM_BanachSpace}), 文献
~\cite{Traub1979,Ypma1982,Smale1986,Wang2000a,Huang2004,Proinov2009,Ferreira2009b}~
分别研究了该迭代法在不同条件下的最优半径及相应局部收敛结果
\footnote{其中文献~\cite{Ypma1982}~拓展了文献~\cite{Traub1979}~的结果;
文献~\cite{Smale1986}~的结果是在解析条件下得到的;
文献~\cite{Huang2004}~的结果是在~H\"{o}lder~条件下得到的;
而文献~\cite{Wang2000a,Proinov2009,Ferreira2009b}~
分别在不同的更一般的条件下统一了
文献~\cite{Traub1979,Ypma1982,Smale1986}~的结果.}.
\item[2)]
半局部收敛性:这种类型在不知方程解~$x^*$~存在的情形下,
根据~$F$~在~某一近似初始点~$x_0$~的局部条件来研究有关迭代法的收敛性质,
一般包括收敛判据、收敛速度以及以~$x_0$~为中心的收敛球和解的唯一性球
\footnote{这里的收敛球和解的
唯一性球与局部收敛分析中的相应概念不同,
文献~\cite{Huang2004}~给出了局部收敛分析的有关定义.}.
\item[3)]
全局收敛性:有别于前面两种,
这种类型研究当~$F$~满足某些适当的~(全局)~条件时,
可以保证取定义域内任意一点作为初始点时
都可收敛到方程的某个解~(有解存在时).
同伦延拓法,线性搜索法和信赖域法是常用的全局化方法,
详见文献~\cite{Ortega1970,Dennis1996,Deuflhard2004}.
\end{itemize}

事实上, 式~(\ref{it:NM_BanachSpace})~只是一种形式记号,
当应用~Newton~法对具体的非线性方程组进行求解时,
实际是求解如下线性方程组:
\begin{equation}
\label{eq:NewtonEquation} F'(x_k)\Delta x_k = - F(x_k),\quad k =
0,1,2,\ldots.
\end{equation}
求得上述方程组的精确解~$\Delta x_k$~后, 由~$x_{k+1} = x_k + \Delta
x_k$~进行迭代修正. 从数值计算的角度看,
Newton~法~(\ref{it:NM_BanachSpace})~具有收敛快的优点,
在实际计算时每一步运算只与前一步有关, 误差不传播且是自校正的. 因此,
在理论和实际应用上都是一种重要的方法. 但是, Newton~法亦有其不足,
例如, 设~$\mathbb{X}=\mathbb{Y} = \CS^n$, 那么
在每步计算中都要计算~$n^2$~个分量偏导数值和~$n$~个分量函数值,
并求一次矩阵的逆, 运算量较大. 为此, Newton~法有不少改进的算法.

这些修正的~Newton~法统称为~Newton~型迭代法, 针对不同的问题背景,
应用相应的~Newton~型迭代法.
常见的有以下几种类型~(详细论述可见文献~\cite{Ortega1970,Traub1982,Deuflhard2004}):
\begin{itemize}
\item
一般的~Newton~法~(初始点~$x_0$~ 给定):
$$
F'(x_k) \Delta x_k = -F(x_k),\ x_{k+1} = x_k + \Delta x_k, \quad k =
0,1,\ldots.
$$
\item
简化~Newton~法:
这种变形的~Newton~法将每步计算~$F'(x_k)$~改为固定的~$F'(x_0)$~(初始点~$x_0$~
给定):
$$
F'(x_0) \Delta x_k = -F(x_k),\ x_{k+1} = x_k + \Delta x_k,\quad k =
0,1,\ldots.
$$
这样, 每步只需要计算~$n$~个分量函数, 但这种迭代法只有线性收敛.
\item
~Newton~类法: 当所要处理的方程组维数较大时,
很难求解~Newton~方程~(\ref{eq:NewtonEquation})~的精确解~$\Delta
x_k$, 进而无法得到精确的~Jacobi~矩阵~$F'(x_{k+1})$.
为了能够应用~Newton~法较好的处理这种情况,
得到了如下的一种变形~Newton~法~(初始点~$x_0$~ 给定):
\begin{equation}
\label{it:NML_general} M(x_k)\Delta x_k = -F(x_k),\ x_{k+1} = x_k +
\Delta x_k,\ \ k = 0,1,\ldots,
\end{equation}
其中~$M(x_k)$~是近似于~$F'(x_k)$~的矩阵.
\item
非精确~Newton~法: 同样考虑大规模方程组情形,
相比于~Newton~类法用一个近似的~Jacobi~矩阵来代替~$ F'(x_k)$,
若考虑在求解~Newton~方程~(\ref{eq:NewtonEquation})~时,
不去求其精确解而只需要求满足某种条件的近似解来作为迭代修正,
这样便得到如下的变形~Newton~法~(初始点~$x_0$~ 给定):
\begin{equation}
\label{it:INM_BanachSpace} F'(x_k) \Delta x_k = -F(x_k) + r_k,\
x_{k+1} = x_k + \Delta x_k,\quad k = 0,1,\ldots,
\end{equation}
其中~$r_k \in \mathbb{Y}$~一般应满足~$\|r_k\|/\|F(x_k)\|\leqslant
\eta_k,\ k= 0,1,\ldots$~, $\{\eta_k\}$~满足~$0\leqslant \eta_k <1$,
可能与~$x_k$~有关, 为控制序列,
用来控制求方程~(\ref{eq:NewtonEquation})~的解的精确程度.
显然令~$\eta_k \equiv 0$~时得到一般的~Newton~法.
\item
拟~Newton~法:Newton~法的主要缺点之一是每步都要计算导数~$F'(x)$~的值,
当分量函数较复杂时计算很不方便,
拟~Newton~法是针对这一缺点提出的算法,
其核心是用通过计算函数值来代替导数以避免求导:
$$
J_k \Delta x_k = -F(x_k), \ J_{k+1} = J_k + \Delta J_k,\quad k =
0,1,\ldots,
$$
其中~$J_k$~是近似于~$F'(x_k)$~的矩阵.不同的~$\Delta J_k$~选择,
可以得到不同的迭代法. 例如,当取$\Delta J_k = F(x_{k+1})\tran{\Delta
x_k}/(\tran{\Delta x_k}\Delta x_k)$, 则得到~Broyden~法。
\item
~Gauss-Newton~法:
这种变形的~Newton~法主要应用于求解非线性最小二乘~(约束/无约束)~问题,
迭代格式为:
\begin{equation}
\label{it:GNM_BanachSpace} \|F'(x_k)\Delta x_k + F(x_k)\| = \min,\
x_{k+1} = x_k + \Delta x_k,\quad k = 0,1,\ldots.
\end{equation}
\end{itemize}


除上述几类重要的~Newton~型迭代法外, 还有几类高阶的~Newton~变形法.
如~Halley~法, 迭代格式为:
\begin{equation}
\label{it:HM_BanachSpace} x_{k+1} = x_k - [\IO -
L_F(x_k)]^{-1}F'(x_k)^{-1}F(x_k), \quad k=0,1,\ldots,
\end{equation}
及~Euler/Chebyshev~法, 迭代格式为:
\begin{equation}
\label{it:EM_BanachSpace} x_{k+1} = x_k - [\IO +
L_F(x_k)]F'(x_k)^{-1}F(x_k), \quad k=0,1,\ldots,
\end{equation}
其中~$L_F(x)=\frac{1}{2}F'(x)^{-1}F''(x)F'(x)^{-1}F(x)$.



为统一研究这两种迭代法的收敛性, Guti\'{e}rrez~ 和
~Hern\'{a}ndez~在文献
~\cite{Gutierrez1997a}~中提出了如下的~Halley-Euler~迭代族法:
\begin{equation}
\label{it:HEFM_BanachSpace} x_{\alpha,k+1} = x_{\alpha,k} -
\left\{\IO + \frac{1}{2}L_F(x_{\alpha,k})[\IO - \alpha
L_F(x_{\alpha,k})]^{-1}\right\}F'(x_{\alpha,k})^{-1}F(x_{\alpha,k}),
\quad k=0,1,\ldots,
\end{equation}
其中~$\alpha \in [0,1]$~及~$L_F(x) =
F'(x)^{-1}F''(x)F'(x)^{-1}F(x)$. 显然, $\alpha =
0$~时~Halley-Euler~迭代族法 ~(\ref{it:HEFM_BanachSpace})~
变为~Euler~法~(\ref{it:EM_BanachSpace}); 当~$\alpha =
\frac{1}{2}$~时~Halley-Euler~迭代族法 ~(\ref{it:HEFM_BanachSpace})~
变为~Halley~法~(\ref{it:HM_BanachSpace});当~$\alpha = 1$~时,
Halley-Euler~迭代族法
~(\ref{it:HEFM_BanachSpace})~变为快速~Halley~法.


关于~Newton~法~(\ref{it:NM_BanachSpace})~
收敛的性质研究主要有以下两个方向:
\begin{itemize}
\item[1)]
Kantorovich~型收敛理论:理论上,
Newton~法收敛性的一个最重要收敛结果是被称为
~Newton-Kantorovich~半局部收敛定理~\cite{Kantorvich1982}.
该定理在理论和应用上都是相当重要的, 它是解方程算法现代研究的起点.
大量的收敛结果都是基于所谓的~Kantorovich~型条件而得到的,
例如,\cite{Ortega1970,Rokne1972,Rall1974,GraggTapia1974,Deuflhard1979,
Ypma1982,Huang1993,Gutierrez1997a,Wang1999,Gutierrez2000,Ezquerro2002}.

\item[2)]
Smale~点估计理论:该理论是由~Smale~于~1986~年提出,由~$\alpha-$理论
和~$\gamma-$理论组成。在~$\alpha-$理论中,
假设~$F$~在初始点~$x_0$~是解析的,
给出了基于如下三个不变量的收敛判据~\cite{Smale1986}:
\begin{equation*}
\begin{cases}
\alpha(F,x_0) = \beta(F,x_0)\gamma(F,x_0),\\
\beta(F,x_0) = \|F'(x_0)^{-1}F(x_0)\|,\\
\gamma(F,x_0) = \sup\limits_{k \geqslant 2} \left\|\displaystyle
\frac{1}{k!} F'(x_0)^{-1}F^{(k)}(x_0)^{-1}\right\|^{\frac{1}{k-1}}.
\end{cases}
\end{equation*}
而~$\gamma-$理论则研究了算子~$F$~在解析条件下的局部收敛性。
定理~\ref{th:SmaleGammaTh}~称为~$\gamma-$定理。
王兴华等人改进并完善了~Smale~点估计理论
~(见文献~\cite{WangLi2001}~及其所列文献). 值得指出的是,
王兴华引入了~$\gamma$~条件
并在此基础上系统建立了~Smale~原先在解析条件下的全部结果
(见文献~\cite{WangHan1997b}~及其所列文献).
\end{itemize}

\begin{theorem}[{\cite[$\gamma-$定理]{Smale1986}}]
\label{th:SmaleGammaTh} 设~$F: \CS^n \to
\CS^n$~是解析的。设~$\zeta$~为~$F$~的一个零点且~$F'(\zeta)^{-1}$~存在。
若~$z \in \CS^n$~满足
\begin{equation}
\label{radius:gamma_AZ} \|z - \zeta\| < \frac{3-\sqrt{7}}{2
\gamma(f,\zeta)},
\end{equation}
则~$z$~为~$F$~关于~$\zeta$~的一个近似零点,即
\begin{equation}
\label{ApproximateZero} \|\zeta - z_k\| \leq
\left(\frac{1}{2}\right)^{2^{k} - 1} \|\zeta - z\|, \quad k = 0, 1,
2, \ldots,
\end{equation}
其中~$\{z_k\}$~为~Newton~法~$(\ref{it:NM_BanachSpace})$~以初始点$z_0
= z$~进行迭代所产生的序列。
\end{theorem}

对于其他的~Newton~型迭代法亦有很多研究结果是建立上述两种条件下而得到的,
例如,关于非精确~Newton~法~(\ref{it:INM_BanachSpace})~研究有~
\cite{ChenLi2006,Guo2007,LiShen2008,ShenLi2009,ShenLi2010,Ferreira2011c},
关于~Gauss-Newton~法~(\ref{it:GNM_BanachSpace})~的研究有~
\cite{DedieuShub2000,DedieuKim2002,Chen2008,LiHuWang2010,Ferreira2011b,XuLi2008}。







Halley~法~(\ref{it:HM_BanachSpace})~和~Euler~法
~(\ref{it:EM_BanachSpace})~是求解非线性方程
~(\ref{eq:NonlinearOperatorEquation})~的~Newton~法的两种重要高阶的迭代法,
而~Newton-Kantorovich~型条件是研究~Newton~法收敛性的重要方向, 因而,
对于~Halley~法和~Euler~法的收敛性,
亦有很多收敛结果是在~Newton-Kantorovich~型条件下得到的,
如~\cite{Candela1990a,Candela1990b,Ezquerro2005,Gutierrez1997b,YeLi2006,YeLiShen2007}.





对于一个迭代法,研究其收敛速度对实际计算是重要的.
为刻画收敛速度,本文引入~$Q$~收敛阶和~$R$~收敛阶.
对于二者关系的详细论述可见文献~\cite{Potra1989,Jay2001,Ortega1970,Rheinboldt1998}.

\begin{definition}[$Q$~收敛阶]
\label{def:Q-orderConv} 设序列~$\{x_k\}$~收敛到~$x^*$~. 如果存在~$q
\geqslant 1$~及常数~$c \geqslant 0$~和~$N \geqslant 0$~使得当~$k
\geqslant N$~时有~$ \|x^* - x_{k+1}\| \leqslant c\|x^* - x_k\|^q$,
则称序列~$\{x_n\}$~具有~$Q$~收敛阶至少为~$q$.特别地,当~$q =
2$~时称为~(至少)~$Q$~平方收敛,$q = 3$~时称为~(至少)~$Q$~立方收敛.
\end{definition}

\begin{definition}[$R$~收敛阶]
\label{def:R-orderConv} 设序列~$\{x_k\}$~收敛到~$x^*$~.
如果存在~$\tau > 1$~及常数~$c \in (0,\infty)$~和~$\theta \in
(0,1)$~使得对所有 ~$n \in \mathbb{N}$~有~$\|x_k - x^*\| \leqslant c
\theta^{\tau^k}$, 则称~$\{x_k\}$~具有~$R$~收敛阶至少为~$\tau$.
\end{definition}







\section{矩阵函数}

矩阵函数~$f:\CS^{n\times n} \to \CS^{n\times n}$~有多种等价定义,
下面所给出的定义是基于~Jordan~典范形而得到的.

对于任意的矩阵~$A \in \CS^{n\times n}$~, 设其~Jordan~典范形为
\begin{equation}
\label{eq:JorCanForm} Z^{-1}AZ = J = \diag(J_1,J_2,\ldots,J_s),
\end{equation}
其中~$Z\in\CS^{n\times n}$~是非奇异的,
$$
J_k = J_k(\lambda_k) = \left[\begin{array}{cccc} \lambda_k & 1 & & \\
 & \lambda_k & \ddots & \\
 & & \ddots & 1 \\
 & & & \lambda_k \end{array}\right] \in \CS^{m_k\times m_k}, \ \
 m_1 + m_2 + \cdots + m_s = n.
$$
设~$\lambda_1,\ldots, \lambda_r$~为~$A$~的所有不同的特征值,
$n_\ell$为 属于特征值~$\lambda_\ell$~的~Jordan~块的阶数,
称为~$\lambda_\ell$~的次数.

\begin{definition}
\label{def:fun_spectrum} 对于任意函数~$f$, 如果
$$
f^{(j)}(\lambda_\ell), \ \ \ j = 0, \ldots, n_\ell-1,\ \ell =
1,\ldots, r
$$
的值都存在, 则称~$f$~在~$A$~的谱上有定义.
\end{definition}

\begin{definition}
\label{def:MatFun} 设函数~$f$~在矩阵~$A \in \CS^{n\times
n}$~的谱上有定义, 且有~Jordan~典范形~(\ref{eq:JorCanForm}), 则有
\begin{equation}
\label{eq:MatFun} f(A) := Zf(J)Z^{-1} = Z\diag(f(J_k))Z^{-1},
\end{equation}
其中
$$
f(J_k) = \left[\begin{array}{cccc} f(\lambda_k) & f'(\lambda_k) & \cdots & \displaystyle \frac{f^{(m_k-1)}(\lambda_k)}{(m_k-1)!} \\
 & f(\lambda_k) & \ddots & \vdots\\
 & & \ddots & f'(\lambda_k) \\
 & & & f(\lambda_k) \end{array}\right].
$$
\end{definition}

下面的定理给出了矩阵函数的若干基本性质,
更多的其他性质参见~\cite{Higham2008}~或~\cite{HornJohnson1991}.

\begin{theorem}[{\cite[定理~1.13]{Higham2008}}]
\label{th:MatFunProperty} 设函数~$f$~在矩阵~$A\in\CS^{n\times
n}$~的谱上有定义, 则有如下性质:
\begin{enumerate}
\item[\textup{(i)}]
$f(A)$~与~$A$~可交换, 即~$f(A)A = Af(A)$;
\item[\textup{(ii)}]
$f(\tran{A})= \tran{f(A)}$;
\item[\textup{(iii)}]
$f(XAX^{-1}) = Xf(A)X^{-1}$;
\item[\textup{(iv)}]
$f(A)$~的全部特征值分别为~$f(\lambda_\ell)$, 其中~$\lambda_\ell,
\ell = 1,\ldots,n$~为~$A$~的特征值;
\item[\textup{(v)}]
如果~$X$~与~$A$~可交换, 那么~$X$~与~$f(A)$~可交换.
\end{enumerate}
\end{theorem}


\begin{theorem}[{\cite[定理~1.15]{Higham2008}}]
\label{th:MatFun_sum_product} 设函数~$f,g$~在矩阵~$A\in\CS^{n\times
n}$~的谱上有定义.
\begin{enumerate}
\item[\textup{(i)}]
若~$h(t) = f(t) + g(t)$, 则~$h(A) = f(A)+g(A)$;
\item[\textup{(ii)}]
若~$h(t)=f(t)g(t)$, 则~$h(A)=f(A)g(A)$.
\end{enumerate}
\end{theorem}






\subsection{矩阵平方根}

对于任意给定的矩阵~$A\in\CS^{n\times n}$,
若存在矩阵~$X\in\CS^{n\times n}$~使得~$X^2 = A$~成立,
则称~$X$~为~$A$~的一个平方根.
下面的定理给出了矩阵平方根存在性的充要条件.

\begin{theorem}[\cite{HornJohnson1991}]
\label{th:MatSquRoot_Existence} 矩阵~$A\in\CS^{n\times
n}$~存在一个平方根的充要条件是如下定义的递增的整数数列~$d_1,
d_2,\ldots$~没有两项都是相同的奇数:
$$
d_\ell = \dim(\nspace(A^\ell)) - \dim(\nspace(A^{\ell-1})), \ \ \
\ell = 1,2,\ldots.
$$
\end{theorem}

\begin{theorem}[矩阵平方根的分类, \cite{Higham2008}]
\label{th:MatSquRoot_Classification} 设非奇异矩阵~$A\in\CS^{n\times
n}$~的~Jordan~典范形由~$(\ref{eq:JorCanForm})$~给出,
其所有不同的特征值数为~$r$. 如果~$r\leq s$,
那么~$A$~有~$2^r$~准平方根, 由如下式子给出:
$$
X_j = Z \diag(L_1^{(j_1)},L_2^{(j_2)},\ldots,L_s^{j_s})Z^{-1},\ \ \
j = 1, 2, \ldots, 2^r,
$$
其中~$j_k = 1$~或~$2$, 当~$\lambda_\ell = \lambda_k$~时~$j_\ell =
j_k,\ k=1,2,\ldots,s$. 特别地, 如果~$r<s$, 那么~$A$~存在非准平方根,
其形式为
$$
X_j(U) =
ZU\diag(L_1^{(j_1)},L_2^{(j_2)},\ldots,L_s^{j_s})U^{-1}Z^{-1}, \ \ \
j=2^r+1, \ldots, 2^s,
$$
其中~$j_k = 1$~或~$2$, $U$~为任意的与~$J$~可交换的非奇异矩阵,
且对于每一个 ~$j$, 存在~$\ell$~和~$k$, 使得当~$j_\ell\neq
j_k$~时~$\lambda_\ell = \lambda_k$.
\end{theorem}



\begin{theorem}[\cite{Higham2008}]
\label{th:MatSquRoot_Principal} 设矩阵~$A\in\CS^{n\times
n}$~的所有特征值都不属于~$\RS^- := (-\infty,0]$.
则存在唯一的~$A$~的准平方根~$X$,
其所有特征值都属于复平面右半部分~$\{z\in\CS: \Real{z}>0\}$.
此时称~$X$~为矩阵~$A$~的主平方根, 记为~$X := A^{1/2}$.
若~$A$~是实矩阵, 则~$A^{1/2}$~也是实矩阵.
\end{theorem}


下面考虑对非奇异矩阵~$A\in\CS^{n\times
n}$~的主平方根~$A^{1/2}$~的计算问题.
记~$f(A)$~为~$A$~的任一准平方根, 并设~$A$~的~Schur~分解为~$A=QTQ^*$,
其中~$Q$~为酉矩阵而~$T$~为上三角矩阵.
由定理~\ref{th:MatFunProperty}~知~$f(A)=Qf(T)Q^*$,
故为了计算矩阵~$A$~的主平方根, 只要计算上三角矩阵~$T$~的主平方根~$U
= f(T)$~即可. 设~$U=[u_{ij}]_{n\times n}, T=[t_{ij}]_{n\times n}$,
由~$U^2 = T$~可得
\begin{align*}
u_{ii}^2 & = t_{ii},\quad i = 1,2,\ldots,n,\\
(u_{ii}+u_{jj})u_{ij} & = t_{ij} - \sum_{k=i+1}^{j-1}u_{ik}u_{kj},
\quad j > i.
\end{align*}
于是, 有如下计算非奇异矩阵平方根的算法,
该算法由~Bj\"{o}rck~$\&$~Hammarling~于~\cite{Bjorck1983}~得到.

\begin{algorithm}[h!]
\floatname{algorithm}{算法}
\caption{计算矩阵平方根的~Schur~法~\cite{Bjorck1983}}
\label{al:MatSquRoot_SchurMethod} 给定非奇异矩阵~$A\in\CS^{n\times
n}$, 本算法通过~Schur~分解来计算~$A$~的主平方根~$A^{1/2}$.
\newcounter{newlist}
\begin{list}{\arabic{newlist}.}{\usecounter{newlist}
\setlength{\rightmargin}{0em}\setlength{\leftmargin}{1.2em}}
\item
计算矩阵~$A$~的~Schur~分解~$A = QRQ^*$;
\item
计算矩阵~$U$~各对角元素的主平方根~$u_{ii} = t_{ii}^{1/2},\ i = 1,
\ldots, n$;
\item
依次计算矩阵~$U$~的非对角元:
$$
u_{ij} = \displaystyle
\frac{t_{ij}-\displaystyle\sum_{k=i+1}^{j-1}u_{ik}u_{kj}}{u_{ii}+u_{jj}},
\quad j = 2,3,\ldots,n,\ i = j-1,j-2,\ldots,1;
$$
\item
计算~$X = QUQ^*$.
\end{list}
\end{algorithm}

算法~\ref{al:MatSquRoot_SchurMethod}~的总计算量为~$28\frac{1}{3}n^3$~flops,
其中~Schur~分解的计算量为~$25n^3$~flops,
$U$~的计算量为~$\frac{1}{3}n^3$~flops, $X$~的计算量为~$3n^3$~flops.

当~$A$~是实矩阵时,
Higham~\cite{Higham1987}~推广了算法~\ref{al:MatSquRoot_SchurMethod}~而得到了如下算法.

\begin{algorithm}[h!]
\floatname{algorithm}{算法}
\caption{计算矩阵平方根的实~Schur~法~\cite{Higham1987}}
\label{al:MatSquRoot_RealSchurMethod} 给定矩阵~$A\in\RS^{n\times
n}$, 其所有特征值都不属于~$\RS^- := (-\infty,0]$.
本算法通过实~Schur~分解来计算~$A$~的主平方根~$A^{1/2}$.
\begin{list}{\arabic{newlist}.}{\usecounter{newlist}
\setlength{\rightmargin}{0em}\setlength{\leftmargin}{1.2em}}
\item
计算矩阵~$A$~的实~Schur~分解~$A = QR\tran{Q}$, 其中~$R$~是~$m\times
m$~的块矩阵.
\item
计算矩阵~$U$~各对角块的主平方根:当~$R_{ii}=[r_{ij}]_{1\times1}$~时,
$U_{ii} = R_{ii}^{1/2}$;当~$R_{ii}=[r_{ij}]_{2\times2}$~时,
$$
U_{ii} = \left[\begin{array}{cc} \alpha +
\displaystyle\frac{1}{4\alpha}(r_{11}-r_{22}) &
\displaystyle\frac{1}{2\alpha}r_{12} \\
\displaystyle\frac{1}{2\alpha}r_{21} & \alpha -
\displaystyle\frac{1}{4\alpha}(r_{11}-r_{22})
\end{array}\right],
$$
其中
\begin{equation*}
\alpha = \left\{
\begin{array}{ll}
\displaystyle \left(\frac{|\theta|+(\theta^2 +
\mu^2)^{1/2}}{2}\right)^{1/2}, & \theta \geq 0,\\
\displaystyle
\frac{\mu}{2\left(\displaystyle\frac{|\theta|+(\theta^2 +
\mu^2)^{1/2}}{2}\right)^{1/2}}, & \theta <0,
\end{array}
\right.
\end{equation*}
$$
\theta = \frac{r_{11}+r_{12}}{2},\quad \mu =
\frac{\left(-(r_{11}-r_{22})^2-4r_{21}r_{22}\right)^{1/2}}{2}.
$$
\item
依次通过计算如下的方程而得到矩阵~$U$~的非对角块~$U_{ij}$:
$$
U_{ii}U_{ij} + U_{ij}U_{jj} = R_{ij} - \sum_{k=i+1}^{j-1}
U_{ik}U_{kj}, \quad j = 2,3,\ldots,m,\ i = j-1,j-2,\ldots,1.
$$
\item
计算~$X = QU\tran{Q}$.
\end{list}
\end{algorithm}



下面考虑应用~Newton~法来计算~$X^2=A$.
设~$Y$~为该矩阵方程的一个近似解, 并记~$X = Y+E$, 则
$$
A = (Y+E)^2 = Y^2 + YE + EY + E^2.
$$
去掉上式中的~$E^2$~项后即可得如下的~Newton~法~($X_0$~给定):
\begin{equation}
\label{it:NM_MatSquRoot_original} X_{k+1} = X_k + E_k, \quad k = 0,
1, 2, \ldots,
\end{equation}
其中~$E_k$~为如下~Sylvesterh~方程的解:
$$
X_k E_k + E_k X_k = A - X^2_k.
$$
一般地, 通过上述~Newton~法来计算矩阵的平方根所需要的计算成本比
算法~\ref{al:MatSquRoot_SchurMethod}~或
~\ref{al:MatSquRoot_RealSchurMethod}~要高很多. 但是,
下面的结果可以改进这个不足.

\begin{lemma}[{\cite[引理~6.8]{Higham2008}}]
假设~Newton~法~$(\ref{it:NM_MatSquRoot_original})$~
的初始点~$X_0$~与矩阵~$A$~可交换,
且所产生的序列~$\{X_k\}$~是有定义的. 则对于所有的~$k\geq1$,
$X_k$~与~$A$~都是可交换的, 且此时~Newton~法的迭代格式为:
\begin{equation}
\label{it:NM_MatSquRoot} X_{k+1} = \frac{1}{2}(X_k+X_k^{-1}A).
\end{equation}
\end{lemma}

Higham~在文献~\cite{Higham1986}~中得到
了~Newton~法~(\ref{it:NM_MatSquRoot})~在初始点取为~$X_0 =
A$~时的二阶收敛结果. 但需要指出的是,
Newton~法~(\ref{it:NM_MatSquRoot})~的数值稳定性较差
~(详细分析见~\cite[6.4~节]{Higham2008}). 为此, Denman $\&$ Beavers
\cite{Denman1976}~给出了Newton~法~(\ref{it:NM_MatSquRoot})~的
一个对偶形式:
\begin{equation}
\label{it:NM_MatSquRoot_Coupled} \left\{
\begin{array}{ll}
\displaystyle X_{k + 1} = \frac{1}{2}(X_k + Y_k^{-1}), & X_0 =
A, \\
\displaystyle Y_{k + 1} = \frac{1}{2}(Y_k + X_k^{-1}), & Y_0 = \I.
\end{array} \right.
\end{equation}
注意到, (\ref{it:NM_MatSquRoot_Coupled})~是数值稳定的
~(详见~\cite[6.4~节]{Higham2008}), 且保持了~Newton~法的二阶收敛性.
特别地, 这种处理方法将在计算矩阵~$p\, (> 2)$~次根中起到重要作用.




Note that Newton's method (\ref{it:NM}) and Halley's method
(\ref{it:HM}) are special cases as (dual) Pad\'{e} family of
iterations which have recently received particular interest for
computing both the principal $p$th root of a complex number and a
matrix, see for example
\cite{Gomilko2012,Laszkiewicz2009,Zietak2013,Iannazzo2008}.



\subsection{矩阵~$p$~次根}

给定矩阵$A \in \CS^{n\times n}$及任意的整数$p \geq 2$,如果存在矩阵
$X \in \CS^{n \times n}$使得$X^p = A$,那么称$X$为$A$的一个$p$次根。


Given a square matrix , a matrix is called a th root of $A$ if
 for any integer . If $A$ has no eigenvalues on $\RS^-$,
the closed negative real axis, there exists a unique principal $p$th
root of $A$, denoted by $A^{1/p}$, which in turn has eigenvalues in
the segment $\{z: -\pi/p < \arg(z) < \pi/p\}$ \cite[Theorem
7.2]{Higham2008}.


\begin{theorem}[矩阵~$p$~次根的存在性, {\cite{Psarrakos2002}}]
\label{th:MatpthRoor_existence} 定义如下的整数列~$\{d_k\}$:
$$
d_k = \dim(\nspace(A^k)) - \dim(\nspace(A^{k-1})),\quad
k=1,2,\ldots.
$$
矩阵~$A\in\CS^{n\times n}$~存在~$p$~次根的充要条件是
对于任一整数~$\nu\geq0$,
数列~$\{d_k\}$~中至多只有一个元素处于~$p\nu$~与~$p(\nu+1)$~之间.
\end{theorem}



\begin{theorem}[矩阵~$p$~次根的分类~\cite{Smith2003}]
设非奇异矩阵~$A\in\CS^{n\times
n}$~的~Jordan~典范形由~$(\ref{eq:JorCanForm})$~给出,
其所有不同的特征值数为~$r$. 如果~$r\leq s$, 那么~$A$~存在~$p^r$~
个~$p$~次根, 由下式给出:
$$
X_j = Z \diag(f_{j_1}(J_1),f_{j_2}(J_2), \ldots,
f_{j_s}(J_s))Z^{-1},\quad j = 1,2,\ldots, p^r,
$$
其中~$j_k \in \{1,2,\ldots,p\}, k=1,2,\ldots,r$, 当~$\lambda_i =
\lambda_k$~时有~$j_i = j_k$. 特别地, 若$r < s$,
则~$A$~存在非~$A$~的函数的~$p$~次根, 由下式给出:
$$
X_j(U) = ZU\diag(f_{j_1}(J_1), f_{j_2}(J_2), \ldots, f_{j_s}(J_s)),
\quad j = p^r+1, \ldots, p^s,
$$
其中~$j_k \in \{1,2,\ldots,p\}, k=1,2,\ldots,r$,
$U$为任意与~$J$~可交换的非奇异矩阵, 对于每一个~$j$, 存在~$i$~和~$k$,
使得当~$\lambda_i = \lambda_k$~时仍有~$j_i \neq j_k$.
\end{theorem}


\begin{theorem}[{\cite[定理~7.2]{Higham2008}}]
\label{th:MatPricipalpthRoot} 设矩阵~$A\in\CS^{n\times
n}$~没有属于~$\RS^- := (-\infty,0]$~的特征值,
则~$A$~存在唯一的~$p$~次根~$X$, 其所有的特征值均属于集合
$$
\left\{z\in \CS: -\frac{\pi}{p} < \arg(z) < \frac{\pi}{p}\right\}.
$$
此时, 称~$X$~为矩阵~$A$~的主~$p$~次根, 并记为~$X = A^{1/p}$.
若~$A$~是实矩阵, 则~$A^{1/p}$~亦是实矩阵.
\end{theorem}

类似于矩阵平方根的情形, 计算一个给定矩阵的
主~$p$~次根通常有直接法和迭代法两种途径. 关于直接法, Smith
\cite{Smith2003}~将算法~\ref{al:MatSquRoot_RealSchurMethod}~
推广至主~$p$~次根的情形.

关于迭代法, 首先考虑~Newton~法. 类似于矩阵平方根的情形,
给定矩阵~$A\in\CS^{n\times n}$, 应用~Newton~法对矩阵方程~$X^p-A =
0,\, (p>2)$~进行求解时,
可通过迭代格式~(\ref{it:NM_MatSquRoot_original})~实现,
此时~$E_k$~通过如下的广义~Sylvester~方程得到:
$$
\sum_{\ell=1}^p X_k^{p-\ell} E_k X_k^{\ell-1} = A - X_k^p.
$$
特别地, 同矩阵平方根的情形一样, 当初始点~$X_0$~与~$A$~可交换时,
可得知对任意的~$k\geq 1$, $X_k$~均与~$A$~可交换.
于是得到如下计算矩阵主~$p$~次根的简化~Newton~法:
\begin{equation}
\label{it:NM_MatpthRoot_original} X_{k+1} = \frac{1}{p}[(p - 1)X_k +
X_k^{1-p}A], \quad X_0A = AX_0.
\end{equation}

显然, 当初始点取~$X_0 = A$~且~$A$~为正定矩阵时,
Hoskins~\cite{Hoskins1979}
证明了~Newton~法~(\ref{it:NM_MatpthRoot_original})~是二阶收敛的.
进一步, 当初始点取~$X_0 = A$~且~$A$~为一般的矩阵时,
Smith~\cite{Smith2003}
证明了~Newton~法~(\ref{it:NM_MatpthRoot_original})~仍然是二阶收敛的.
当初始点取~$X_0 = \I$~时,Iannazzo
\cite{Iannazzo2006}~得到了如下的收敛性结果:

\begin{theorem}[\cite{Iannazzo2006}]
\label{th:Conv_NM_Ian2006} 给定矩阵~$A\in\CS^{n\times
n}$,设其谱为~$\sigma(A)$。若~$\sigma(A) \subset \MCE_1$, 其中
\begin{equation}
\label{set:ConvReg_NM_Ian2006} \MCE_1 := \{z\in\CS: \Real{z}
>0, |z| \leq 1\},
\end{equation}
则以~$X_0 = \I$~为初始点的
~Newton~法~$(\ref{it:NM_MatpthRoot_original})$~
所产生的矩阵序列~$\{X_k\}$~收敛于矩阵~$A$~的主~$p$~次根~$A^{1/p}$.
\end{theorem}

之后,Iannazzo~在~\cite{Iannazzo2008}~得到了一个新的收敛域:

\begin{theorem}[\cite{Iannazzo2008}]
\label{th:Conv_NM_Ian2006} 给定矩阵~$A\in\CS^{n\times
n}$,设其谱为~$\sigma(A)$。若~$\sigma(A) \subset \MCE_2$, 其中
\begin{equation}
\label{set:ConvReg_NM_Ian08} \MCE_2 := \left\{z\in\CS: |z|\leq 2,
|\arg(z)|< \frac{\pi}{4}\right\},
\end{equation}
则以~$X_0 = \I$~为初始点的
~Newton~法~$(\ref{it:NM_MatpthRoot_original})$~
所产生的矩阵序列~$\{X_k\}$~收敛于矩阵~$A$~的主~$p$~次根~$A^{1/p}$.
\end{theorem}

最近,Guo \cite{Guo2010}~进一步得到了一个更好的收敛域:

\begin{theorem}[\cite{Guo2010}]
\label{th:Conv_NM_Guo2010} 给定矩阵~$A\in\CS^{n\times
n}$,设其谱为~$\sigma(A)$。若~$\sigma(A) \subset
\MCE_3$~且零特征值~$($若存在$)$~是半单的,其中
\begin{equation}
\label{set:ConvReg_NM_Guo2010} \MCE_3 := \{z\in\CS: |z-1| \leq 1\},
\end{equation}
则以~$X_0 = \I$~为初始点的
~Newton~法~$(\ref{it:NM_MatpthRoot_original})$~
所产生的矩阵序列~$\{X_k\}$~收敛于矩阵~$A$~的主~$p$~次根~$A^{1/p}$,
且是二阶收敛的.
\end{theorem}


然而, 在实际计算中~Newton~法~(\ref{it:NM_MatpthRoot_original})~
的数值稳定性并不好, 详细的稳定性分析可见~\cite[3.2~节]{Smith2003}.
于是,基于~Denman $\&$
Beavers~在计算矩阵平方根时所给出的具有数值稳定性的
~Newton~法~(即对偶~Newton~法~(\ref{it:NM_MatSquRoot_Coupled})),
Iannazzo \cite{Iannazzo2006}提出了如下的~Newton~迭代格式来计算矩阵
主~$p$~次根:
\begin{equation}
\label{it:NM_MatpthRoot_Coupled} \left\{
\begin{array}{ll}
\displaystyle X_{k + 1} = X_k \left(\frac{(p-1)\I + N_k}{p}\right),
& X_0 = \I, \\
\displaystyle N_{k + 1} = \left(\frac{(p-1)\I + N_k}{p}\right)^{- p}
N_k, & N_0 = A.
\end{array} \right.
\end{equation}
称~(\ref{it:NM_MatpthRoot_Coupled})~为对偶~Newton~法,
该迭代格式的一个优点是具有很好的数值稳定性。显然,当~$N_k \to
\I$~时~$X_k \to A^{1/p}$。 在~\cite{Iannazzo2008}~中,
Iannazzo~给出了计算矩阵主~$p$~次根的算法~
\ref{al:MatpthRoot_SchurNewton_Ian08}:

\begin{algorithm}[h!]
\floatname{algorithm}{算法}
\caption{计算矩阵主~$p$~次根的~Schur-Newton~法~\cite[算法~
3]{Iannazzo2008}} \label{al:MatpthRoot_SchurNewton_Ian08}
给定矩阵~$A\in\RS^{n\times n}$, 其所有特征值都不属于~$\RS^- :=
(-\infty,0]$. 给定整数~$p\geq 2$, 则存在整数~$k_0 \geq
0$~及奇数~$q$~使得~$p = 2^{k_0}q$.
本算法通过实~Schur~分解和对偶~Newton~法~(\ref{it:NM_MatpthRoot_Coupled})~
来计算~$A$~的主~$p$~次根~$A^{1/p}$.
\begin{list}{\arabic{newlist}.}{\usecounter{newlist}
\setlength{\rightmargin}{0em}\setlength{\leftmargin}{1.2em}}
\item
计算矩阵~$A$~的实~Schur~分解~$A = QR\tran{Q}$.
\item
若~$q=1$,令~$k_1 = k_0$; 若~$q\neq1$,则选取~$k_1\geq k_0$~使得
存在正数~$s$~使任意~$A$~的特征值~$\lambda$~满足
$$
s \lambda^{1/{2^{k_1}}} \in \left\{z\in \CS: \left|z -
\frac{6}{5}\right| \leq \frac{3}{4}\right\}.
$$
\item
通过算法~\ref{al:MatSquRoot_RealSchurMethod}~计算~$B =
R^{1/{2^{k_1}}}$.
\item
若~$q=1$, 则令~$X = QB\tran{Q}$;若~$q\neq 1$,
则通过对偶~Newton~法~(\ref{it:NM_MatpthRoot_Coupled})~ 来计算~$C =
(B/s)^{1/q}$~并令~$X = Q(Cs^{1/q})^{2^{k_1-k_0}}\tran{Q}$.
\end{list}
\end{algorithm}


Guo $\&$ Higham
\cite{GuoHigham2006}~给出了一种含参数的对偶~Newton~法:
\begin{equation}
\label{it:NM_MatpthRoot_ParCoupled} \left\{
\begin{array}{ll}
\displaystyle X_{k + 1} = \left(\frac{(p+1)\I -
N_k}{p}\right)^{-1}X_k,
& X_0 = c\I, \\
\displaystyle N_{k + 1} = \left(\frac{(p+1)\I - N_k}{p}\right)^p
N_k, & \displaystyle N_0 = \frac{1}{c^p}A.
\end{array} \right.
\end{equation}
显然,当~$N_k \to \I$~时~$X_k \to
A^{1/p}$。此外,也给出了计算矩阵主~$p$~次根的算法
~\ref{al:MatpthRoot_ParSchurNewton_GuoHig06}:

\begin{algorithm}[h!]
\floatname{algorithm}{算法}
\caption{计算矩阵主~$p$~次根的含参数~Schur-Newton~法~\cite[算法~
3.3]{GuoHigham2006}} \label{al:MatpthRoot_ParSchurNewton_GuoHig06}
给定矩阵~$A\in\RS^{n\times n}$, 其所有特征值都不属于~$\RS^- :=
(-\infty,0]$. 给定整数~$p\geq 2$, 则存在整数~$k_0 \geq
0$~及奇数~$q$~使得~$p = 2^{k_0}q$.
本算法通过实~Schur~分解和含参数~Newton~法~
(\ref{it:NM_MatpthRoot_ParCoupled})~
来计算~$A$~的主~$p$~次根~$A^{1/p}$.
\begin{list}{\arabic{newlist}.}{\usecounter{newlist}
\setlength{\rightmargin}{0em}\setlength{\leftmargin}{1.2em}}
\item
计算矩阵~$A$~的实~Schur~分解~$A = QR\tran{Q}$.
\item
若~$q=1$,令~$k_1 = k_0$; 若~$q\neq1$,则选取~$k_1\geq k_0$~使得
$|\lambda_1/\lambda_n|^{1/2^{k_1}}\leq 2$, 其中~$\lambda_1,\ldots,
\lambda_n$~为~$A$~的特征值且满足~$|\lambda_n|\leq \cdots \leq
|\lambda_1|$,
当~$\lambda_\ell$~不全是实数时,重新选取~$k_1$~使得对任意的~$\ell
\in \{1,2,\ldots,n\}$~都有
$$
\arg(\lambda_\ell^{1/2^{k_1}}) \in
\left(-\frac{\pi}{8},\frac{\pi}{8}\right).
$$
\item
通过算法~\ref{al:MatSquRoot_RealSchurMethod}~计算~$B =
R^{1/{2^{k_1}}}$.
\item
若~$q=1$, 则令~$X = QB\tran{Q}$;若~$q\neq 1$, 则先选取参数~$c$,
再通过含参数对偶~Newton~法~ (\ref{it:NM_MatpthRoot_ParCoupled})~
来计算~$C = B^{1/q}$~并令~$X = QC^{2^{k_1-k_0}}\tran{Q}$.
\end{list}
\end{algorithm}


Halley~法是计算矩阵主~$p$~次根的另一种重要的迭代法。
类似于~Newton~法,若初始点~$X_0$~与矩阵~$A$~可交换,
则可得如下的计算矩阵主~$p$~次根的简化~Halley~法:
\begin{equation}
\label{it:HM_MatpthRoot_original} X_{k+1} = X_k\left((p+1)X_k^p +
(p-1)A\right)^{-1} \left((p-1)X_k^p + (p+1)A\right), \quad AX_0 =
X_0A.
\end{equation}
特别地,初始点取为~$X_0 =
\I$~是研究时主要考虑的情形,如~\cite{Iannazzo2008,Guo2010,Lin2010}。
关于~Halley~(\ref{it:HM_MatpthRoot_original})~法的收敛性, Iannazzo
\cite{Iannazzo2008}~得到了如下的结果:

\begin{theorem}[\cite{Iannazzo2008}]
\label{th:Conv_HM_Ian08} 给定矩阵~$A\in\CS^{n\times
n}$,设其谱为~$\sigma(A)$。若~$\sigma(A) \subset \{z\in \CS:
\Real{z}>0\}$, 则以~$X_0 = \I$~为初始点的
~Halley~法~$(\ref{it:HM_MatpthRoot_original})$~
所产生的矩阵序列~$\{X_k\}$~收敛于矩阵~$A$~的主~$p$~次根~$A^{1/p}$.
\end{theorem}


Guo~在~\cite{Guo2010}~中进一步证明了当 ~$\sigma(A) \subset \{z\in
\CS:
\Real{z}>0\}$~时~Halley~法~(\ref{it:HM_MatpthRoot_original})~是三阶收敛的。
应用在\cite{Iannazzo2006}中对Newton法稳定性的分析方法可知,
Halley~法~(\ref{it:HM_MatpthRoot_original})~同样在数值计算中是不稳定的。
故类似于~Newton~法,可引入如下的具有数值稳定的对偶形式的~Halley~法:
\begin{equation}
\label{it:HM_MatpthRoot_Coupled} \left\{
\begin{array}{ll}
\displaystyle X_{k+1} = X_k\left((p+1)\I + (p-1)N_k\right)^{-1}
\left((p-1)\I + (p+1)N_k\right),
& X_0 = \I, \\
\displaystyle N_{k + 1} = N_k\left((p+1)\I + (p-1)N_k\right)^{-1}
\left((p-1)\I + (p+1)N_k\right), & \displaystyle N_0 = A.
\end{array} \right.
\end{equation}
显然,当~$N_k \to \I$~时~$X_k \to
A^{1/p}$。算法~\ref{al:MatpthRoot_SchurHalley_Ian08}~
给出了应用对偶~Halley~法~(\ref{it:HM_MatpthRoot_Coupled})~
来计算~$A$~的主~$p$~次根~$A^{1/p}$。

\begin{algorithm}[h!]
\floatname{algorithm}{算法}
\caption{计算矩阵主~$p$~次根的~Schur-Halley~法~\cite[算法~
4]{Iannazzo2008}} \label{al:MatpthRoot_SchurHalley_Ian08}
给定矩阵~$A\in\RS^{n\times n}$, 其所有特征值都不属于~$\RS^- :=
(-\infty,0]$. 给定整数~$p\geq 2$, 则存在整数~$k_0 \geq
0$~及奇数~$q$~使得~$p = 2^{k_0}q$.
本算法通过实~Schur~分解和对偶~Halley~法~(\ref{it:HM_MatpthRoot_Coupled})~
来计算~$A$~的主~$p$~次根~$A^{1/p}$.
\begin{list}{\arabic{newlist}.}{\usecounter{newlist}
\setlength{\rightmargin}{0em}\setlength{\leftmargin}{1.2em}}
\item
计算矩阵~$A$~的实~Schur~分解~$A = QR\tran{Q}$.
\item
若~$q=1$,令~$k_1 = k_0$; 若~$q\neq1$,则选取~$k_1\geq k_0$~使得
存在正数~$s$~使任意~$A$~的特征值~$\lambda$~满足
$$
s \lambda^{1/{2^{k_1}}} \in \left\{z\in \CS: \left|z -
\frac{8}{5}\right| \leq 1\right\}.
$$
\item
通过算法~\ref{al:MatSquRoot_RealSchurMethod}~计算~$B =
R^{1/{2^{k_1}}}$.
\item
若~$q=1$, 则令~$X = QB\tran{Q}$;若~$q\neq 1$,
则通过对偶~Halley~法~(\ref{it:HM_MatpthRoot_Coupled})~ 来计算~$C =
(B/s)^{1/q}$~并令~$X = Q(Cs^{1/q})^{2^{k_1-k_0}}\tran{Q}$.
\end{list}
\end{algorithm}




The results concerning convergence of these two methods have
recently been studied under the assumption that the eigenvalues of
$A$ are all lie in some region, see for example
\cite{GuoHigham2006,Guo2010,Iannazzo2006,Iannazzo2008,Lin2010,Zietak2013}.
In particular, Guo in \cite{Guo2010} shown that the matrix sequence
$\{X_k\}$ generated by Newton's method (\ref{it:NM}) starting from
the identity matrix converges to the principal $p$th root of $A$ if
all of whose eigenvalues lie in the set $\MCE_1 :=\{z\in\CS:
|z-1|\leq 1\}$. Iannazzo obtained in \cite{Iannazzo2008} that the
matrix sequence $\{X_k\}$ generated by Halley's method (\ref{it:HM})
starting also from the identity matrix converges to $A^{1/p}$ for
each $A$ having eigenvalues in the set $\MCE_2 := \{z\in\CS: \Real z
> 0\}$.
$\I$







\subsection{代数~Riccati~方程}






























\section{论文的组织}
